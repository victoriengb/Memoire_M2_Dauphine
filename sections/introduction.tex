\section*{Introduction}
Au sein de l'industrie informatique moderne, les chaînes d'intégration et de
déploiement continues (CI/CD) jouent un rôle crucial dans la livraison rapide et
fiable des logiciels. Cependant, avec l'évolution rapide des technologies et des
pratiques de développement, ces chaînes CI/CD peuvent rapidement devenir
obsolètes, posant des défis significatifs pour les entreprises d'un point de vue
de l'efficacité opérationnelle et de la sécurité des systèmes d'information. Ce
mémoire traitera la problématique suivante : \textbf{Dans quelle mesure la
modernisation technologique de la chaîne CI/CD, couplée à une stratégie de
\textit{knowledge management} permet de transformer l'obsolescence technique en
levier stratégique ?} 

Ce mémoire propose une stratégie de migration de la chaîne CI/CD vers des
solutions plus modernes en passant de Jenkins vers GitHub Actions et en faisant
une transition de serveurs on-premises vers un cluster de serveurs Kubernetes.
La migration technologique d'un microservice sera présentée dans cette étude. La
solution comportera également une stratégie d'accompagnement au changement et de
\textit{knowledge management} pour pérenniser la connaissance au sein de
l'entreprise et favoriser son adaptabilité.

\subsection*{Contexte}
% Provide context and background information about the topic
Le service Alltra est en charge d'une application éponyme de gestion
documentaire de transactions commerciales de matières premières. L'application
est composée d'une quarantaine de microservices mais repose sur une
infrastructure CI/CD vieillissante (Jenkins/XLDeploy). Cela pose des problèmes
d'automatisation car, aujourd'hui, les microservices sont en partie déployés
manuellement allongeant considérablement la mise en production. De plus, du fait
d'une configuration vieillissante (Jenkinsfile / Groovy) la correction des
erreurs de build est plus compliquée. Certains plugins Jenkins présentent en
outre des failles de sécurité (Common Vulnerabilities and Exposures ou CVE) et
ne sont plus maintenus par la communauté.

L'usage de GitHub Actions permettra de rationaliser la chaîne CI/CD. Aujourd'hui
la redondance Jenkins/GitHub Actions crée un surcoût injustifié quant aux coûts
des licences GitHub et au maintien des serveurs Jenkins tandis que GitHub et
GitHub Actions forment une solution intégrée. De plus, d'un point de vue
technique les compétences de l'équipe Alltra sont plus alignées avec l'usage de
GitHub Actions (configuration YAML) qu'avec celui de Jenkins (configuration
Groovy). La configuration YAML, déclarative, permet de faire abstraction de la
logique sous-jacente de la configuration des pipelines CI/CD tandis que la
configuration Groovy, impérative, force la programmation de la logique
sous-jacente et augmente ainsi le risque d'erreurs et la complexité de la
configuration. 

Enfin, la connexion au serveur Jenkins depuis le serveur GitHub crée une
complexité injustifiée dans la chaîne CI/CD. L'adoption de GitHub Actions
supprime ce besoin de connexion et regroupe l'hébergement du code et son
déploiement. Les plugins sur GitHub Actions sont appelés des actions. La Société
Générale dispose d'actions internes sur GitHub Actions qui offrent des
fonctionnalités d'automatisation CI/CD. Une gouvernance quant à la gestion du
cycle de vie de ces actions sera nécessaire et discutée au cours de ce mémoire.
La Société Générale peut également bénéficier d'un écosystème d'actions
officielles venant de comptes certifiés (Verified Creators) contrairement aux
plugins communautaires de Jenkins qui ne sont pas certifiés. Ma stratégie de
migration consiste à exploiter les actions certifiées de GitHub Actions pour
automatiser la chaîne CI/CD et à utiliser les plugins internes de la Société
Générale pour ses besoins spécifiques.

Par ailleurs, les serveurs on-premises actuellement supportés seront décommissionnés
cette année et doivent donc impérativement être remplacés. Je propose de migrer
vers un cluster Kubernetes. Ce choix stratégique répond à trois enjeux :

\begin{enumerate}
    \item \textbf{L'Immutabilité :} Contrairement aux serveurs on-premises
    historiques sujets à des conflits de configuration au fil des déploiements,
    l'usage de conteneurs éphémères assure une isolation de l'environnement
    d'exécution en isolant chaque microservice et limite ainsi les effets de
    bord. Les conteneurs permettent également de conserver la même configuration
    sans impacter l'infrastructure matérielle sous-jacente. Cela limite les
    besoins de maintenance au niveau de la configuration et simplifie la
    maintenance de l'infrastructure CI/CD. Les conteneurs garantissent également
    une configuration dite iso-prod assurant que les dépendances du code source
    en environnement de développement soient identiques à celles en environnement
    de production.
    \item \textbf{L'Élasticité :} La capacité d'\textit{autoscaling} permet
    d'adapter dynamiquement les ressources à la charge réelle, assurant la
    performance sans intervention humaine.
    \item \textbf{L'Optimisation des ressources :} En lançant plusieurs
    microservices sur un même serveur à travers des conteneurs sur un cluster
    Kubernetes au lieu de les déployer isolément sur des serveurs distincts, on
    optimise l'usage des ressources et on réduit les coûts de maintenance des
    serveurs existants ainsi que la complexité de l'infrastructure informatique.
\end{enumerate}

Enfin, nous démontrerons que la réponse à l'obsolescence n'est pas uniquement
technologique, mais culturelle. La migration vers des standards modernes impose
une \textbf{conduite du changement} rigoureuse pour accompagner la montée en
compétence des équipes vers une culture \textit{DevOps} autonome.

De plus, afin d'éviter que la nouvelle solution ne devienne la dette technique
de demain, nous aborderons les stratégies de \textit{knowledge management}. À
travers l'adoption de pratiques \textit{InnerSource} et la documentation
\textit{as-code}, l'objectif est de pérenniser le savoir-faire et de garantir la
maintenabilité corrective et évolutive à long terme de la chaîne de livraison.