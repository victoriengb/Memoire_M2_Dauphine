\section*{Introduction}
Au sein de l'industrie informatique moderne, les chaînes d'intégration et de
déploiement continues (CI/CD) jouent un rôle crucial dans la livraison rapide et
fiable des logiciels. Cependant, avec l'évolution rapide des technologies et des
pratiques de développement, ces chaînes CI/CD peuvent rapidement devenir
obsolètes, posant des défis significatifs pour les entreprises d'un point de vue
de l'efficacité opérationnelle et de la sécurité des systèmes d'information.
Nous allons voir dans ce mémoire comment gérer l'obsolescence des chaînes CI/CD
au sein d'une entreprise en explorant les stratégies adoptées pour maintenir
leur pertinence et leur efficacité dans un environnement en constante évolution.
Une solution dans le cadre de la Société Générale CIB sera proposée à travers la
migration de la chaîne CI/CD vers des solutions plus modernes en passant de
Jenkins vers Github Actions et en passant de serveurs on-premises vers un
cluster de serveurs Kubernetes pour un microservice. Dans un premier temps, nous
présenterons le contexte et les enjeux liés à l'obsolescence des chaînes CI/CD,
avant d'explorer les différentes approches pour y faire face. Nous allons
notamment présenter les stratégies de remédiation de l'obsolescence ainsi que
des outils et technologies émergentes qui peuvent être intégrés dans les chaînes
CI/CD pour améliorer leur résilience et leur adaptabilité. Enfin, nous
conclurons en soulignant l'importance de la gestion proactive de l'obsolescence
des chaînes CI/CD pour assurer la compétitivité et la sécurité des entreprises
dans le paysage technologique actuel.

\subsection*{Contexte}
% Provide context and background information about the topic
Le service Alltra est en charge d'une application éponyme de gestion
documentaire de transactions commerciales de matières premières. L'application
est composée d'une quarantaine de microservices mais repose sur une
infrastructure ci/cd vieillissante (Jenkins/XLDeploy). Cela pose des problèmes
d'automatisation car aujourd'hui, les microservices sont en partie déployés
manuellement allongeant considérablement la mise en production. De plus, du fait
d'une configuration vieillissante (Jenkinsfile / Groovy) la correction des
erreurs de build est plus compliquée. Certains plugins Jenkins présentent en
outre des failles de sécurité (Common Vulnerabilities and Exposures ou CVE) et
ne sont plus maintenus par la communauté.

Passer à l'usage de Github Actions permettra de rationaliser la chaîne ci/cd.
Aujourd'hui le doublon Jenkins/Github Actions crée un surcoût injustifié quant
aux coûts des licences Github et au maintien des serveurs Jenkins tandis que
Github et Github Actions forment une solution intégrée. De plus, d'un point de vue
technique les compétences de l'équipe Alltra sont plus alignées avec l'usage de
Github Actions, plus aisée (configuration YAML), qu'avec Jenkins (configuration
Groovy). La configuration YAML, déclarative, est plus aisée à lire et donc à
maintenir que la configuration Groovy. Enfin la connexion au serveur Jenkins
depuis le serveur Github crée une complexité injustifiée (à préciser) dans la
chaîne ci/cd. Le passage à Github Actions supprime ce besoin de connexion et
regroupe l'hébergement du code et son déploiement. Les plugins sur Github
Actions sont appelés des actions. La Société Générale dispose de d'actions
internes sur Github Actions qui offrent des fonctionnalités d'automatisation
ci/cd. Une gouvernance quant à la gestion du cycle de vie de ces actions sera
nécessaire et discuté au cours de ce mémoire. La Société Générale peut également
bénéficier d'un écosystème d'actions officielles venant de comptes certifiés
(Verified Creators) contrairement aux plugins communautiares de Jenkins qui ne
sont pas certifiés. Ma stratégie de migration consiste à
exploiter les actions certifiés de Github Actions pour automatiser la chaîne
ci/cd et à utiliser les plugins internes de la Société Générale pour ses besoins
spécifiques.

Aussi, les serveurs on-premises actuellement supporté seront décommissionnés
cette année et doivent donc impérativement être remplacés. Je propose de passer
à un cluster Kubernetes. Ce choix stratégique répond à trois enjeux :

\begin{enumerate}
    \item \textbf{L'Immutabilité :} Contrairement aux serveurs on-premises
    historiques sujets à des conflits de configuration au fil des déploiements,
    l'usage de conteneurs éphémères assure une isolation de l'environnement
    d'exécution en isolant chaque microservices et limite ainsi les effets de
    bord. Les conteneurs permettent également de conserver la même configuration
    sans impacter l'infrastructure matérielle sous-jacente. Cela limite les
    besoins de maintenance au niveau de la configuration et simplifie la
    maintenance de l'infrastructure ci/cd. Les conteneurs garantissent également
    une configuration dîte iso-prod assurant que les dépendances du code source
    en environnement de développement soit identique à celles en environnement
    de production.
    \item \textbf{L'Élasticité :} La capacité d'\textit{autoscaling} permet
    d'adapter dynamiquement les ressources à la charge réelle, assurant la
    performance sans intervention humaine.
    \item \textbf{L'Optimisation des ressources :} En lançant plusieurs
    microservices sur un même serveur à travers des conteneurs sur un cluster
    Kubernetes au lieu de les déployer isolément sur des serveurs distincts, on
    optimise l'usage des ressources et on réduit les coûts de maintenance des
    serveurs existants ainsi que la complexité de l'infrastructure informatique.
\end{enumerate}

Enfin, nous démontrerons que la réponse à l'obsolescence n'est pas uniquement
technologique, mais culturelle. La migration vers des standards modernes impose
une \textbf{conduite du changement} rigoureuse pour accompagner la montée en
compétence des équipes vers une culture \textit{DevOps} autonome.

De plus, afin d'éviter que la nouvelle solution ne devienne la dette technique
de demain, nous aborderons les stratégies de \textbf{gestion des connaissances}
(\textit{Knowledge Management}). À travers l'adoption de pratiques
\textit{InnerSource} et la documentation \textit{as-code}, l'objectif est de
pérenniser le savoir-faire et de garantir la maintenabilité à long terme de la
chaîne de livraison.

%\subsection*{Problem Statement} Define the main problem or question your essay
% addresses

%\subsection*{Objectives} Outline the goals and objectives of your work

%\subsection*{Scope} Describe the scope and limitations of your essay

%\subsection*{Organization} Briefly describe how the essay is structured and
% organized